%%%%%=== Introduction ===%%%%%
\chapter{引言}
\thispagestyle{fancy}
\section{研究背景与意义}

我国幅员辽阔,随着人民生活水平日益提高,人们开始注重精神生活的丰富,年轻人更是勇于探索各处未知山野。然而由于各种因素,报纸新闻中也不乏"驴友"迷失,救援队伍耗费大量人力物力展开救援的报道。此外,一些特殊工作岗位人员,如野外考古学家,地质工作者等因为天气、通讯故障等原因也有可能发生人员失踪的事故。通常来说,野外迷失人员精神和生理压力骤升,体能消耗加速,尤其是遭遇自然灾害后,人体可能已经遭受损伤,因此野外搜救就是在和时间赛跑,分秒必争。但是阻碍救援因素却不少,恶劣的天气,通讯故障,交通不便等都是致命因素。搜救行动往往牵动大量人力物力。传统的救援方式是地面搜索,在某些特殊环境比如山林或者山谷,救援人员和设备运输就会受阻,导致搜救效率就十分低下。


如果能够从空中进行搜索,救援行动就会简便高速很多。随着近几年无人机(Unmanned Ariel Vehicles, UAV)行业的快速发展,这已经成为可能。无人机能够搭载图像采集设备以及各种传感器,可以将空中的俯瞰视野实时传回地面,同时,无人机行动敏捷,快速。基于以上考虑,通过多传感器数据融合实现无人机的一些自主行为(如自主避障,自动路径规划,自动返航等),同时,利用机载计算设备,结合目标检测识别技术,进行待救援人员的检测与定位,能够大幅度减少野外搜救耗时间,为关键的生命救援争取时间,所以本系统具有很广的应用前景。




\section{难点与创新点}

为了达到野外搜救的目的,系统设计需要实现无人目标搜寻、路径规划、自主避障以及移动降落功能。针对上述四个功能需求,本系统实现有以下难点:
\begin{enumerate}[1.]
\item 野外搜救场景较为复杂,目标表观姿态较多,识别目标具有很大难度;
\item 目前尚无公开的无人机视角下人体数据库,需要自己建立无人机视角人体数据库;
\item 无人机航拍图像分辨率较高,无法实时对视频进行目标检测,同时,单独使用光学相机不能保证系统全天候运作;
\item 无人机自主避障技术尚无成熟的技术框架;
\item 为了保证搜救行动的效率,需要保证无人机能够自主降落在移动平台上。
\end{enumerate}



针对上述难点的分析与研究,本文成功设计了一种面向野外搜救的智能无人机系统。其创新点在于:
\begin{enumerate}[1.]
\item 针对本系统中,野外搜救环境较为复杂的特点,使用目前综合检测性能最好的基于深度卷积神经网络的目标检测算法SSD算法作为基础算法,并针对本系统的应用场景进行了优化改进,提高了检测性能;
\item 建立了无人机视角人体数据库,便于对改进后的SSD算法进行重新训练,提高此应用场景下的检测性能;
\item 使用热成像仪和光学相机两路视频源进行目标检测,使用热成像仪得到目标的候选区域,再进行光学相机目标检测,既提高了搜索效率,又保证了本系统能全天候正常工作;
\item 针对无人机自主避障这一技术难题,使用双目相机进行障碍物位置测算,并通过合理的路径规划策略规避障碍物;
\item 针对无人机移动平台自主降落这一技术难题,使用AprilTag视觉基准系统辅助无人机进行自主降落。
\end{enumerate}

\section{本文研究工作与内容安排}
本文致力于研究一套面向野外搜救的智能无人机系统。本文研究工作与内容安排如下:

\subsection{研究工作}

本文主要研究工作如下:
\begin{enumerate}[1)]
\item 无人机目标搜索。采用双源视频流分析方法,天空端使用机载计算设备进行热成像仪红外图像处理,根据红外图像的特点得到目标候选区域。地面端通过高清图传接收器获取实时的光学相机视频流。考虑到无人机视角下人体表观变化巨大,传统的基于手工设计特征的目标检测方法效果不佳,因此基于目前最先进的基于深度学习的检测算法SSD,并进行了相应的改进。此外,针对野外搜救这一应用场景,本系统构建了一个无人机视角下的人体数据集,并利用该数据集训练改进SSD算法模型。
\item 无人机自主避障。使用双目相机,根据双目视差原理获取障碍物的点云图,借助机器人操作系统ROS(Robot Operating System)下的导航工具包,实现了基于代价地图的无人机自主避障。
\item 无人机自主移动降落。本系统还实现了基于增强现实的基准视觉系统AprilTag的识别算法,并将其应用于无人机相对位姿获取。无人机根据得到的6个DOF(Degree of Freedom)坐标信息进行自身的姿态调整,从而实现精准移动降落。
\item 完整软硬件系统的搭建与测试。基于以上三个关键技术,本论文实现了面向野外搜救的智能无人机系统。系统通过大量实验调试来提高整体稳定性。此外,还编写了用于必要交互的移动应用程序以及方便地面工作人员可视化搜救结果的PC端图像界面程序。
\end{enumerate}

\subsection{内容安排}
根据本论文工作的内容,本论文的各章节安排与主要内容如下:
第一章主要介绍了无人机野外搜救系统的研究背景与意义,最后给出论文的主要工作以及整体结构;

第二章介绍系统需求与方案设计;

第三章介绍无人机搜救系统的硬件组成。包括无人机平台和各种使用到的传感器,并对各自的特点及参数进行介绍;

第四章介绍无人机搜救系统的软件设计。包括无人机地面目标搜索的软件实现,基于双目视觉的无人机自主避障技术的软件实现以及基于视觉辅助的无人机自主移动降落技术的软件实现;

第五章介绍系统测试结果,包括软硬件系统各个功能模块以及完整系统的实验结果以及分析;

第六章进行总结与展望。总结工作的主要贡献和不足,并对下一步的研究方向进行讨论。


