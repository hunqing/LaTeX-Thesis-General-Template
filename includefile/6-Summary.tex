\chapter{总结与展望}
\thispagestyle{fancy}

本文面向野外搜救应用,设计并实现了一套野外无人机搜救系统。针对搜救无人机系统的核心任务和实际需求,设计了系统的软硬件架构。整体硬件架构包括无人机系统平台、双目视觉模块,云台相机模块(包括光学相机和红外相机)、核心计算设备,安卓远程设备,地面工作站的设计与实现,并成功搭建了系统原型,针对野外搜救任务中的三个重要功能:目标搜索、自主避障和移动降落,进行了软件算法设计和验证实验。在目标检测方面,采用了红外图像辅助的基于深度学习目标检测方法,并构建了一个无人机多视角的人体检测数据集,实验结果表明,深度卷积神经网络目标检测方法可以从数据中学习到相比手工设计更能表达目标的特征,也因此获得更高的准确率。在无人机自主避障和路径规划方面,系统通过双目点云估计障碍深度信息,障碍信息通过更新代价地图的方式更新避障路径,从而指导无人机避开障碍。在无人机移动降落方面,利用增强现实技术中的AprilTag视觉基准系统,通过对降落标志的荷载进行编码处理来最大限度的减少虚警。并通过求解PnP问题,获取无人机相机相对于标志的位姿,并以此作为无人机位姿调整的依据,精确引导无人机降落在标志上。实验结果表明利用移动降落可以达到$10cm$以内的降落精度,移动可以降落在$2\sim3m/s$的移动平台上。在系统搭建上,本系统采用天空端和地面站两线工作模式。自主避障,红外目标候选区域提取以及自主返航移动降落均由机载计算设备Manifold上完成。地面端主要由遥控器、安卓手机和高性能笔记本电脑组成。为了完成必要的交互以及充当天空端机载计算设备与地面笔记本的通信中介,开发了一个移动应用(APP)。为了可视化搜索结果,在笔记本电脑上开发了基于Qt5的图形界面程序。天空端和地面端程序均使用多线程以及模块化编程,各个模块之间不存在耦合,十分方便单独测试。多次飞行试验表明搭建的系统原型稳定性较高,可以成功完成搜救试验。

本论文主要将无人机搜救工作的三个关键技术进行研究并将其以功能模块的形式集成到一个搜救系统原型中,因此后续仍有很大的发展空间。随着传感器技术的发展,传感器的种类将会越来越多,因此,无人机搜救系统可以考虑搭载更多的传感器,进行多元信息融合从而提高搜救效率。另外,目前目标检测算法是基于深度学习的框架,但是为了充分发掘深度网络模型在特定应用场景下的潜力,仍需要大量的自建数据。目前本系统使用的自建数据集样本数量偏少,后续工作应该包括扩大数据集。此外,单就“人”这一类别而言,姿态为平躺或俯卧,蜷缩等姿态的样本仍然十分有限,因此可以借助机器学习中生成式对抗网络(Generative Adversarial Nets, GAN)进行此类样本的自动生成,进一步扩充数据集。第三点就是增加更多地面站功能。现在的无人机飞控系统通常会支持MAVLink协议,这样就使得地面站功能(比如航点飞行,热点跟随)可以很方便实现。实际实施搜救时,使用到更多地面站功能,就能更方便地规划整个搜救任务,进一步提高搜救效率。
