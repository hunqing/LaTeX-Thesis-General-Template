%%==============================%%
%%=======填写中文摘要===========%%
%%==============================%%
\CNabstract
{
\thispagestyle{empty}
我国幅员辽阔,野外灾害频发,搜救难度极大。通常来说,野外搜救就是和时间赛跑,分秒必争。搜救行动往往牵动大量人力物力。传统的地面救援方式效率低下,极易错过黄金搜救时间。近年来无人机(Unmanned Ariel Vehicles, UAV)的快速发展,使快速搜救成为可能。UAV搭载图像采集设备以及各种传感器,可将空中的俯瞰视野实时传回地面。此外,UAV敏捷,快速,并且可以以低操作成本执行难以由人类执行的动作。这些特点使得UAV很适合应用到搜救领域。目前,在搜救领域中,所用UAV大多只进行图像采集,其自主性没有得到很好的发挥。因此,本文研究了野外搜救无人机系统中的关键技术:自主避障、路径规划和自动返航;同时,结合计算机视觉中的目标检测识别技术,进行待救援人员的检测与定位。通过模块化的设计,旨在充分利用无人机平台资源的基础上,大幅度减少野外搜救耗时,为关键的生命救援争取时间。

在无人机目标搜索算法设计上,采用结合红外和光学图像的深度卷积神经网络检测方法。先使用阈值分割和形态学处理方法获取红外图像中的显著区域,再通过红外图像与光学图像配准,找到光学影像中的候选区域。基于候选区域,使用改进的深度卷积神经网络模型SSD算法进行实时目标检测。改进的SSD算法通过自建数据集进行模型训练。

关于无人机自主避障,采用了基于双目视觉的避障方案。主要是利用双目点云得到无人机前方障碍物的深度信息。然后根据障碍点云信息更新代价地图,进而不断更新避障路线,从而实现自主避障和路径规划。

关于自主移动降落,引入了增强现实技术中的视觉基准系统AprilTag。通过编解码AprilTag标志,检测算法可以大大降低虚警,同时由于编码时引入容错机制,漏检率也保持在一个很低的水平。然后通过求解PnP问题,无人机能够自主动态调整其位姿,从而实现精准移动降落。在静止降落的基础上,利用分阶段处理,实现了降落在移动的平台(如车顶,卡车后部)上。

基于上述基础关键技术,本文创新性的实现了野外无人机搜救系统的原型搭建。系统采用模块化开发,模块之间耦合度低,可移植性强。配套的软件程序包括安卓应用程序以及基于Qt5的多线程图形界面程序。软硬件完整系统实验结果表明,实现的野外无人机搜救系统原型稳定性良好,可以完成搜救任务,为打造集成度更高功能更强效率更高的搜救系统提供了良好的基础。
\thispagestyle{empty}
}

%%==============================%%
%%=======填写中文关键词===========%%
%%==============================%%
%%注意: 每个关键词之间用“;”分开,最后一个关键词不打标点符号
\CNkeywords{\thispagestyle{empty}
    野外搜救; 无人机; 目标检测; 自主避障; 移动降落}