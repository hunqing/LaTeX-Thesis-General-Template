\usepackage{silence}
\WarningFilter{caption}{Unsupported document class}%%消除无关警告,caption宏包本身问题
\usepackage{tocloft,enumerate,paralist,amsmath,lmodern,tikz,array}
\usepackage[fancyhdr,fntef,space]{ctexcap}
\usepackage[top=25mm,bottom=20mm,left=30mm,right=30mm,includefoot,xetex]{geometry}
\usepackage[labelsep=quad,belowskip=5pt,aboveskip=5pt]{caption}
\usepackage{underscore}
\usepackage{graphics}
\usepackage{xcolor}
\usepackage{subcaption}
\usepackage[super,square]{natbib}
\usetikzlibrary{shapes,arrows,chains}


\renewcommand\theequation{
  \thechapter-\roman{equation}
}
%% 页眉相关设置
% \makeatletter
% \newif \if@frontmatter   \@frontmatterfalse
% \let \LaTeXfrontmatter = \frontmatter
% \let \LaTeXmainmatter  = \mainmatter
% \renewcommand {\frontmatter} {\@frontmattertrue  \LaTeXfrontmatter}
% \renewcommand {\mainmatter}  {\@frontmatterfalse \LaTeXmainmatter}

% \pagestyle{fancy}
%   \fancyhf{}
%   \fancyhead[RE]{\if@frontmatter\else\small \CAST@value@titlemark\fi}
%   \fancyhead[LO]{\if@frontmatter\else\small \leftmark\fi}
%   \fancyhead[LE,RO]{\if@frontmatter\else\small ~\thepage~\fi}
%   \fancyfoot[C]{\if@frontmatter\small ~\thepage~\fi}
%   \renewcommand{\headrulewidth}{\if@frontmatter 0pt\else .7pt\fi}%
% \fancypagestyle{plain}{%
%   \pagestyle{fancy}
% }
% \makeatother


\renewcommand\theequation{\arabic{chapter}-\arabic{equation}}
\renewcommand\thetable{\arabic{chapter}-\arabic{table}}
\renewcommand\thefigure{\arabic{chapter}-\arabic{figure}}

% \usepackage{subfigure}
% \usepackage{titlesec}
% \usepackage{titletoc}

% \titlecontents{capter}  
%               []  
%               {\heiti \zihao{4}}%  
%               {\contentslabel{2.5mm}}%  
%               {}%  
%               {\titlerule*[0.5pc]{$\cdot$}\contentspage\hspace*{3cm}}

%\hypersetup{citecolor=magenta,linkcolor=blue}%设置超链接颜色,打印版本可使用下一行设置

\hypersetup{allcolors=black}
\hypersetup{pdfencoding=auto,pdfauthor=Qingqing\ Cao,pdfsubject=bsthesis}

%%设置图表标题格式
\DeclareCaptionFont{hei}{\heiti\zihao{-4}}
\DeclareCaptionFont{song}{\songti\zihao{-4}}
\captionsetup[table]{font=hei}
\captionsetup[figure]{font=song}

%%%%%设置目录样式
\setlength\cftaftertoctitleskip{0ex}
\renewcommand\cftchapafterpnum{\vskip5pt}
\renewcommand\cftsecafterpnum{\vskip3pt}
\renewcommand\cftsubsecafterpnum{\vskip2pt}
\setlength{\cftsecindent}{1.85em}
\setlength{\cftsubsecindent}{3.7em}
\setlength{\cftsubsubsecindent}{3.7em}

\renewcommand\cftdot{…}
\renewcommand\cftdotsep{0}
% \setlength{\cftsubsecindent}{3.7em}
% \setlength{\cftsubsubsecindent}{3.7em}

\setlength\cftbeforetoctitleskip{-1cm}

\renewcommand{\cfttoctitlefont}{\hfill\heiti\zihao{3}}%目录标题 黑体小 2 号
\renewcommand{\cftaftertoctitle}{\hfill}

\renewcommand\cftchapfont{\heiti\zihao{4}}%目录内容中章的标题 黑体 4 号
\renewcommand\cftsecfont{\songti\zihao{-4}}%目录中其他内容 宋体小 4 号
\renewcommand\cftsubsecfont{\songti\zihao{-4}}%目录中其他内容 宋体小 4 号
\renewcommand\cftsubsubsecfont{\songti\zihao{-4}}%目录中其他内容 宋体小 4 号

\renewcommand\cftchappagefont{\heiti\zihao{4}}
\renewcommand\cftsecpagefont{\songti\zihao{-4}}%目录页码字体
\renewcommand\cftsubsecpagefont{\songti\zihao{-4}}
\renewcommand\cftsubsubsecpagefont{\songti\zihao{-4}}

%%%%%章节标题样式
\CTEXsetup[format+={\heiti \zihao{-2}},nameformat={},titleformat={},%
number={\arabic{chapter}},afterskip={1.5ex},beforeskip={0ex}]{chapter}% 各章标题 黑体小 2 号
\CTEXsetup[format={\heiti \zihao{-3}\raggedright}]{section}%各节的一级标题 黑体 4 号
\CTEXsetup[format={\heiti \zihao{-4}\raggedright}]{subsection}% 各节的二级标题 黑体小 4 号
\CTEXsetup[format={\heiti \zihao{-4}\raggedright}]{subsubsection}%各节的三级标题 黑体小 4 号
\CTEXsetup[name={附录},number={\Alph{chapter}}]{appendix}
\setcounter{tocdepth}{2}
\setcounter{secnumdepth}{3}


%%修改图表的标题字体
\renewcommand\figurename{\songti\zihao{-4} 图}
\renewcommand\tablename{\heiti\zihao{-4} 表}

\pagestyle{fancy}


\bibliographystyle{abbrv}        % 参考文献样式,  plain,unsrt,alpha,abbrv,chinesebst 等等
\newcommand{\upcite}[1]{\textsuperscript{\cite{#1}}}  %自定义新命令\upcite, 使参考文献引用以上标出现

\newcommand*\circled[1]{\tikz[baseline=(char.base)]{%
\node[shape=circle,draw,inner sep=0.5pt] (char) {#1};}}%圆圈数字①

%%更改公式编号中括号为中文括号
\makeatletter
\def\tagform@#1{\maketag@@@{(\ignorespaces#1\unskip\@@italiccorr)}}
\renewcommand{\eqref}[1]{\textup{{\normalfont(\ref{#1})\normalfont}}}
\makeatother

%%修改脚注样式
\makeatletter%
\long\def\@makefnmark{%
\hbox {{\normalfont \textsuperscript{\circled{\@thefnmark}}}}}%
\makeatother
\makeatletter%
\long\def\@makefntext#1{%
  \parindent 1em\noindent \hb@xt@ 1.8em{\hss \circled{\@thefnmark}}#1}%
\makeatother
\skip\footins=10mm plus 1mm
\footnotesep=6pt
\renewcommand{\footnotesize}{\songti\zihao{5}}
\renewcommand\footnoterule{\vspace*{-3pt}\hrule width 0.3\columnwidth height 1pt \vspace*{2.6pt}}

\graphicspath{{figures/}} % 图片文件路径

\renewcommand{\normalsize}{\zihao{-4}} %正常字号
\setmainfont{Times New Roman}
\setsansfont{Times New Roman}



\newtoks\StudentNumber %学号
\newtoks\miji %密级,空着

\newtoks\Cmajor %学科专业中文名
\newtoks\Emajor %学科专业英文名

\newtoks\Ctitle %中文题目
\newtoks\Etitle %英文题目

\newtoks\Cauthor %作者名字
\newtoks\Eauthor %作者英文名

\newtoks\Esupervisor %指导教师英文名
\newtoks\Csupervisor %指导教师中文名

\newtoks\Eschoolname %学院英文名
\newtoks\Cschoolname %学院中文名

\newtoks\Cdate %日期
\newtoks\Edate %英文日期

\newtoks\CNabstract
\newtoks\CNkeywords

\newtoks\ENabstract
\newtoks\ENkeywords
